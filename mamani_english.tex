\documentclass[margin,line,10pt]{res}

\usepackage{verbatim}
\usepackage[brazilian]{babel}
\usepackage[spanish]{babel}
%\usepackage[latin1]{inputenc}
%\usepackage[portuguese]{babel}
\usepackage{hyperref}
\usepackage{color}
\usepackage{multicol}

\documentclass[margin,line,10pt]{res}

\oddsidemargin -.5in
\evensidemargin -.5in
\textwidth=6.0in
\itemsep=0in
\parsep=0in

\newenvironment{list1}{
  \begin{list}{\ding{113}}{%
      \setlength{\itemsep}{0in}
      \setlength{\parsep}{0in} \setlength{\parskip}{0in}
      \setlength{\topsep}{0in} \setlength{\partopsep}{0in} 
      \setlength{\leftmargin}{0.17in}}}{\end{list}}

\newenvironment{list2}{
  \begin{list}{$\bullet$}{%
      \setlength{\itemsep}{0in}
      \setlength{\parsep}{0in} \setlength{\parskip}{0in}
      \setlength{\topsep}{0in} \setlength{\partopsep}{0in} 
      \setlength{\leftmargin}{0.2in}}}{\end{list}}

\begin{document}

\name{Gerardo Cornelio Mamani Mamani \hspace{8 cm} July 2024 \vspace*{.1in}}

\begin{resume}

\section{\sc Contact Information}
\vspace{.05in}
\begin{tabular} {@{}p{3in}p{4in}}
270 S Russell Street, West Lafayette IN, 47907  & \hspace{1.8cm} {\it E-mail:} gerardocmamani@gmail.com \\      United States    & \hspace{1.8cm} {\it Cell:}  (+1)7655329066, (+51) 967198620    \\     
\end{tabular}


\vspace{0.3cm}

\section{\sc Research and Professional Interests}

I worked in rural development projects of production systems of alpacas, llamas, guinea pigs and cattle with the objective of raising the income levels of farmers. My academic and professional interest is Animal breeding and genomics, Statistics and Animal Reproduction. 
 
\section{\sc Education}

{\bf University of Sao Paulo}, Pirassununga, Sao Paulo, Brazil\\
\vspace*{-.1in}
\begin{list1}
\item[] Ph.D. in Animal Bioscience, October 2018
\begin{list2}
\vspace*{.05in}
\item Thesis: "Association between inbreeding genomics coefficients and productive traits in Nellore cattle and Santa Ines sheep" 
\item Advisor: Prof. Dr. Jose Bento Sterman Ferraz 
\item Committee: Drs. Jose Bento Sterman Ferraz, Joanir Pereira Eler, Julio Cesar de Carvalho Balieiro, Ricardo Vieira Ventura and Victor Breno Pedrosa.
\end{list2}
\vspace*{.05in}
\end{list1}

{\bf University of Nebraska - Lincoln}, Lincoln, NE, USA\\
\vspace*{-.1in}
\begin{list1}
\item[] Visitor Scholar, 4/2017 - 12/2017
\begin{list2}
\vspace*{.05in}
\item Project: "Genomic prediction in Nellore cattle" 
\item Supervisor: Prof. Dr. Gota Morota
\end{list2}
\vspace*{.05in}
\end{list1}

{\bf University National Agrarian La Molina}, Lima, Peru\\
\vspace*{-.1in}
\begin{list1}
\item[] Magister Scientiae in Animal Production, March 2013
\begin{list2}
\vspace*{.05in}
\item Thesis: "Genetic structure and genetic trend of birth weight in alpacas of the Germoplasm Bank of Quimsachata Puno - Peru". 
\item Advisor: Prof. Dr. Gustavo Gutierrez Reynoso 
\item Committee: Drs. Juan Chavez Cossio, Jorge Aliaga and Edwin Mellisho Salas
\end{list2}
\vspace*{.05in}
\end{list1}

{\bf University National of the Altiplano}, Puno, Peru\\
\vspace*{-.1in}
\begin{list1}
\item[] Veterinary Doctor,  Marzo 2005
\begin{list2}
\vspace*{.05in}
\item Tesis: "Relationship between sperm morphology, motility and vigor and the enzymatic activity of GOT and GPT transaminases in alpacas". 
\item Advisor: Prof. Dr. Pedro Coila 
\item Committee: Drs. Maximo Melo, Guido Perez and Zenon Maquera
\end{list2}
\end{list1}

\section{\sc Work \phantom{1cm} Experience}

\vspace{0.5cm}

Purdue University\\
{\bf Project: Integrating genomics, phenotyping, and nutrition strategies to enhance alpaca fiber quality and survival in Arequipa’s highlands}, Peru.\\
\vspace{-.3cm}
\textbf{Position:} Posdoc Assistant Research  \hfill {\bf 02/2024 - current}\\
\vspace{0.3cm}

\vspace{0.5cm}

Instituto Nacional de Innovacion Agraria\\
{\bf Project: Mejoramiento de la Disponibilidad de Material Genetico de Ganado Bovino a Nivel Nacional}, Huaral, Peru.\\
\vspace{-.3cm}
\textbf{Position:} Especialist in Animal Genetic  \hfill {\bf 11/2021 - 05/2023}\\
\vspace{0.3cm}

Instituto Nacional de Innovacion Agraria\\
{\bf Subdireccion de Biotecnologia de la Direccion de Recursos Geneticos y Biotecnologia}, Lima, Peru.\\
\vspace{-.3cm}
\textbf{Position:} Coordinator EEA Puno and EEA Ayacucho \hfill {\bf 02/2020 - 10/2020}\\

Instituto Nacional de Innovacion Agraria\\
{\bf Subdireccion de Biotecnologia de la Direccion de Recursos Geneticos y Biotecnologia}, Lima, Peru.\\
\vspace{-.3cm}
\textbf{Position:} Especialist in Molecular Biology and Genomic \hfill {\bf 11/2019 - 01/2020}\\

Banco de Evaluadores del FONCyT\\
{\bf Agencia Nacional de Promocion Cientifica y Tecnologica }, Buenos Aires, Argentina.\\
\vspace{-.3cm}
\textbf{Position:} Evaluador de Proyectos PICT \hfill {\bf 02/2017}\\

Project: Improving Alpaca Systems Production in Highland of the Central Sierra of Peru\\
{\bf University National of La Molina - VLIR}, Lima - Pasco, Peru\\
\vspace{-.3cm}
\textbf{Position:} Research assistant \hfill {\bf 08/2013 - 08/2014}\\

Elaboration of baseline and technical expedient of livestock development in Orcopampa, Arequipa\\
{\bf IPTIG}, Lima, Peru\\
\vspace{-.3cm}
\textbf{Position:} Consultant  \hfill {\bf 05/2013 - 06/2013}\\

Project: Improvement of the capacity and employment for the alpaca production in the Northeast of Puno, Peru\\
{\bf CECOALP}, Puno, Peru\\
\vspace{-.3cm}
\textbf{Position:} Extension.  \hfill {\bf 04/2010 - 08/2010}\\

Study of baseline and elaboration of the project for the development of South American domestic camelids in Antauta and Ajoyani, Puno.\\
{\bf ONG DESCO}, Puno, Peru\\
\vspace{-.3cm}
\textbf{Position:} Consultant.  \hfill {\bf 03/2010}\\

Project: Development of productive and commercial capacities of the small breeders of alpacas of Manazo and Cabanillas, Puno.\\
{\bf ONG CEDER}, Puno, Peru\\
\vspace{-.3cm}
\textbf{Position:} Extension.  \hfill {\bf 01/2008 - 12/2009}\\

Project: Desarrollo de capacidades productivas y comerciales de los pequenos criadores de alpacas de las comunidades altoandinas de Moquegua y Arequipa.\\
{\bf ONG CEDER - CESVI}, Puno, Peru\\
\vspace{-.3cm}
\textbf{Position:} Extension.  \hfill {\bf 11/2006 - 12/2007}\\

Programa de Apoyo a Campesinos Pastores de Alturas PROALPACA-Huancavelica.\\
{\bf ONG Vecinos Peru}, Ayacucho-Huancavelica, Peru\\
\vspace{-.3cm}
\textbf{Position:} Extension.  \hfill {\bf 11/2005 - 10/2006}\\

Elaboracion de linea de base del Project de desarrollo de capacidades, incremento productivo y mercadeo de leche y derivados de las comunidades campesinas del altiplano de Puno CARITAS- Puno.\\
{\bf ONG DESCO}, Puno, Peru\\
\vspace{-.3cm}
\textbf{Position:} Pollster.  \hfill {\bf 10/2005}\\

Inversiones Doria SRLtda. Santa Lucia - Puno.\\
{\bf Project de Desarrollo Corredor Puno -Cusco}, Puno, Peru\\
\vspace{-.3cm}
\textbf{Position:} Extension.  \hfill {\bf 10/2005}\\

\vspace{0.5cm}


\section{\sc Editorial \phantom{1cm} Activities}

\begin{list7}
\item {\bf \underline{Reviewer}}: Livestock Science (2018) and Spermova Journal (2016).  
\end{list7}
\vspace{0.5cm}

% \section{\sc Papers in development}
% 
% \begin{list1}
% 
% \item [{\bf 5}.]  {\bf \underline{Mamani GC}}, Santana BF, Oliveira G, Mattos E, Eler J, Morota G, Ferraz JBS. Effect of genomic inbreeding en productive traits in Nellore catle.  
% \vspace{0.5cm}
% 
% \item [{\bf 4}.]  {\bf \underline{Mamani GC}}, Santana BF, Alexandre P, Oliveira G, Mattos E, Eler J, Pinto LF, Mourao G, Morota G, Ferraz JB. Effect of genomic inbreeding en productive traits in Santa Ines sheep. 
% \vspace{0.5cm}
% 
% \item [{\bf 3}.]  {\bf \underline{Mamani GC}}, Santana BF, Berton M, Mattos E, Eler J, Morota G, Dias L, Molento M, Ferraz JB. Runs of homozygosity and copy number variation in Thoroughbred horses in Brazil. 
% \vspace{0.5cm}
% 
% \item [{\bf 2}.]  {\bf \underline{Mamani GC}}, Mamani-Cato RH, Gutierrez G, Gutierrez JP. Population structure in the Germoplasm Bank of Alpacas in Peru. 
% \vspace{0.5cm}
% 
% \item [{\bf 1}.]  {\bf \underline{Mamani GC}}, Zimmermann M, Vargas N, Hidalgo V, Gutierrez G. Genetic parameters in growth traits in Guinea Pigs.
% \end{list1}
% 
% \vspace{0.5cm}

\vspace{0.7cm}
\section{\sc Journal Papers}
\vspace{0.9cm}


\section{\sc 2021}

\begin{list1}
\item [{\bf 6}.] {\bf \underline{Mamani GC}}, Gonzales, M.L. Number of progeny number and accuracy of breeding value in productive traits of alpacas. A simulation study. 
\textcolor{blue}{\href{https://revistasinvestigacion.unmsm.edu.pe/index.php/veterinaria/article/view/19083}{{\it Revista de Investigaciones Veterinarias del Peru } }}
\end{list1}

\section{\sc 2020}

\begin{list1}
\item [{\bf 5}.] Bussiman FO, dos Santos BA, Silva BCA, {\bf \underline{Mamani GC}}, Grigoletto L, Pereira GL, Ferraz GC, Ferraz JBS, Mattos EC, Eler JP, Ventura RV, Curi RA, Balieiro JCC. Genome-wide association study: Understanding the genetic basis of the gait type in Brazilian Mangalarga Marchador horses, a preliminary study. 
\textcolor{blue}{\href{https://www.sciencedirect.com/science/article/abs/pii/S1871141319306493}{{\it Livestock Science} }}
\end{list1}

\section{\sc 2019}

\begin{list1}
\item [{\bf 4}.] Cancino-Baier DE, {\bf \underline{Mamani GC}}, Santana BF, Mattos EC, Eler JP, Sainz RD, Tonetto T, Tonetto V, Tonetto F, Quinones JA, Sepulveda NG and Ferraz JBS. Estimation of variance components for carcass traits in Guzerat cattle. 
\textcolor{}{\href{https://www.geneticsmr.com/articles/estimation-variance-components-carcass-and-production-traits-guzerat-cattle}{{\it Genetic and Molecular Research} {\bf 18}.3}}
\end{list1}

\section{\sc 2017}

\begin{list1}
\item [{\bf 3}.]  Santana BF, {\bf \underline{Mamani GC}}, Oliveira G, Castro L, Molento M, Ventura R, and Ferraz JBS. Characterization of runs of homozygosity in a population of Thoroughbred Horses.   
\textcolor{black}{\href{http://www.citec.fatecjab.edu.br/index.php/files/article/view/1161}{{\it Revista Ciencia & Tecnologia}.{\bf v9}.2017}. } 
\vspace{0.5cm}

\item [{\bf 2}.] Santana BF, Fonseca R, Matos M, {\bf \underline{Mamani GC}}, Eler J and Ferraz JBS. Feasibility of using "days to a specific weight" traits in Nellore cattle breeding programs.   
\textcolor{black}{\href{http://www.scielo.br/scielo.php?script=sci_arttext&pid=S1519-99402017000200260}{{\it Revista Brasileira de Saude e Producao Animal}. {\bf 18}.2}}
\end{list1}
\vspace{0.5cm}

\section{\sc 2015}

\begin{list1}
\item [{\bf 1}.]  Vargas A, Gutierrez G. and {\bf \underline{Mamani GC}}.  
 An application of Gibbs sampling for genetic parameters estimation in guinea pigs using MCMCglmm.   
     \textcolor{black}{\href{http://dev.scielo.org.pe/scielo.php?script=sci_arttext&pid=S1609-91172015000200003&lng=en&nrm=iso}{{\it Revista de Investigaciones Veterinarias del Peru}. {\bf 26}.2 } } 
\end{list1}
\vspace{0.5cm}




\vspace{0.5cm}
\section{\sc Proceedings Abstract}
\vspace{0.9cm}

\section{\sc 2024}

\begin{list1}

\item [\bf{27}.] {\bf \underline{Mamani GC}}, Dennis Pilares, Luis P. Sousa, Alejandra Toro-Ospina, Jon Schoonmaker, Julia Bello-Bravo, Jennifer Richardson and Luiz F. Brito.
Implementación de un programa de mejora genética en alpacas mediante simulación. En {\it
III International Seminar on Sustainable Production of South American Camelids: Securing Biodiversity}. 10 - 12 July, Lima - Peru. 

\vspace{0.5cm}

\end{list1}

\section{\sc 2022}

\begin{list1}

\item [\bf{26}.] Leon  S, Mamani Chullo R, Davila J, {\bf \underline{Mamani GC}}, Quilcate C,  and Dipaz-Berrocal D.
Influence of breed, season, and ejaculate number on the sperm quality of breeding bulls in Inia, Peru. En {\it
Proceedings of the Annual Conference of the International Embryo Technology Society}. 16 - 19 January, Lima - Peru. 
\vspace{0.2cm}
\item [\bf{25}.] {\bf \underline{Mamani GC}}, Santana B, Jarquin D. 
Assessing genomic prediction of economic trait in alpacas: a simulation study. En {\it
XII World Congress on Genetics Applied to Livestock Production}. 3 - 8 Julio, Rotterdam - The Netherlands. 
\vspace{0.5cm}

\end{list1}

\section{\sc 2021}

\begin{list1}

\item [\bf{24}.] {\bf \underline{Mamani GC}}, Luis Telo da Gama, Eler JP,  Mattos EC, Gabriela Giacomini G, Oliveira JL, Nunez R, Santana BF, Medeiros GC, Ferraz JBS. Efeito da idade da mãe sobre caracteristicas do bovino Montana Composto Tropical. En {\it XIV Simposio Brasileiro de Melhoramento Animal}, Santa Catarina, Brasil. 
\end{list1}

\section{\sc 2019}

\begin{list1}

\item [\bf{23}.] {\bf \underline{Mamani GC}}, Oliveira Jr GA, Santana BF,  Ventura R, Mattos EC, Eler J, Baruselli PS, Morota G, Ferraz JBS. Depressão endogâmica associada a cada cromossomo no numero de foliculos antrais em novilhas Nelore. En {\it XIII Simposio Brasileiro de Melhoramento Animal}, Salvador, Bahia, Brasil. 
\vspace{0.5cm}

\item [\bf{22}.] Santana BF, {\bf \underline{Mamani GC}}, Fragomeni B, Ventura R, Bussiman F, Mattos EC, Eler J, Ferraz JBS. Comparacao entre dois grupos de touros Nelore com diferentes niveis de prepotencia em relacao a padroes de autozigosidade para a caracteristica peso a desmama. En {\it XIII Simposio Brasileiro de Melhoramento Animal}, Salvador, Bahia, Brasil. 
\vspace{0.5cm}

\item [\bf{21}.] Ferraz JBS, Oliveira Jr. G, Santana BF, Eler J, Oliveira EC, Fukumazu H, Morota G, {\bf \underline{Mamani GC}}. Cattle Inbreeding in Nellore Cattle Estimated By Pedigree and Runs of Homozygosity and Its Effect on Antral Follicle Numbers of Heifers. {\it XXVII Plant and Animal Genome}. San Diego, CA, USA. 
\vspace{0.5cm}

\item [\bf{20}.] Santana B, Fragomeni B, {\bf \underline{Mamani GC}}, Oliveira EC, Eler J, Fukumazu H, Ferraz JBS. Cattle Association between Runs of Homozygosity and Capacity to Breed Uniform Progenies in Beef Cattle Bulls. {\it XXVII Plant and Animal Genome}. San Diego, CA, USA.

\end{list1}

\section{\sc 2018}

\begin{list1}

\item [\bf{19}.] Santana BF, Fragomeni B, {\bf \underline{Mamani GC}},  Mattos E, Eler J, Ferraz JBS.
Association between runs of homozygosity pattern and low variability of progeny’s estimated breeding values in Nellore cattle. October 18, Data Science Day - Yale University, USA. 
\vspace{0.5cm}

\item [\bf{18}.] {\bf \underline{Mamani GC}}, Santana BF, Mattos E, Eler J, Mourao G, Batista Pinto LF, Morota G, Ferraz JBS. Ilhas de homozigose em ovinos Santa Ines.
VII Simposio de Biociencia Animal. September 4, Pirassununga, SP, Brazil. 
\vspace{0.5cm}

\item [\bf{17}.] {\bf \underline{Mamani GC}}, Santana BF, Sartorello GL, Abreu Silva BA, Mattos E, Eler J, Morota G, Ferraz JBS. Economic impact of inbreeding over weaning weight, yearling weight and cow productivity in Nellore. EAAP Annual Meeting 2018, Dubrovnik, Croatia. 
\vspace{0.5cm}

\item [\bf{16}.] {\bf \underline{Mamani GC}}, Santana BF, Mattos E, Eler J, Morota G, Ferraz JBS. Effect of genomic inbreeding in productive traits in Santa Ines sheep.
XXVI Reunion de la Asociacion Latinoamericana de Produccion Animal. Guayaquil, Ecuador. 
\vspace{0.5cm}

\item [\bf{15}.] Santana BF, {\bf \underline{Mamani GC}}, Mattos E, Eler J, Ferraz JBS. 
Estrutura populacional em racas bovinas usando dados genomicos. 
XXVI Reunion de la Asociacion Latinoamericana de Produccion Animal. Guayaquil, Ecuador. 
\vspace{0.5cm}

\item [\bf{14}.] {\bf \underline{Mamani GC}}, Santana BF, Oliveira G, Ventura R, Mattos E, Eler J, Morota G, Ferraz JBS. 
Effect of inbreeding in productive traits in Nellore cattle.
XI World Congress on Genetics Applied to Livestock Production. Auckland, New Zeland. 
\vspace{0.5cm}

\item [\bf{13}.] Santana BF, {\bf \underline{Mamani GC}}, Oliveira G, Ventura R, Mattos E, Eler J, Ferraz JBS. 
Association of levels of homozygosity in Nellore bulls with high and low variability of estimated breeding values of their progenies.
XI World Congress on Genetics Applied to Livestock Production. Auckland, New Zeland. 
\end{list1}

\section{\sc 2017}
\begin{list1}
\item [\bf{12}.] {\bf \underline{Mamani GC}}, Santana BF,  Ventura R, Ferraz JBS. 
Caracterizacao de segmentos de homozigose en bovinos, ovinos, porcinos e equinos.
VI Simposio de Biociencia Animal . September 5, Pirassununga, SP, Brazil. 
\vspace{0.5cm}

\item [\bf{11}.] Santana BF, {\bf \underline{Mamani GC}}, Oliveira G,  Ventura R, Ferraz JBS. 
Efeitos da endogamia sobre a uniformidade de progenie de touros nelore..
VI Simposio de Biociencia Animal. September 5, Pirassununga, SP, Brazil. 
\end{list1}

\section{\sc 2016}
\begin{list1}
\item [\bf{10}.] {\bf \underline{Mamani GC}}, and Santana BF, Aguirre L, Mattos EC, Ferraz JBS. 
Estructura genetica de ovinos Santa Ines por analise de pedigree.
III Simposio de Biociencia Animal. Setiembre 8, Pirassununga, SP, Brazil. 
\end{list1}

\section{\sc 2015}
\begin{list1}
\item [\bf{9}.] Mamani-Cato RH, Huanca T, Condori-Rojas N, {\bf \underline{Mamani GC}}.
Estructura genetica de la poblacion de llamas del Banco de Germoplasma del INIA Peru
Reunion Cientifica Anual de la Asociacion Peruana de Produccion Animal. Agosto 7-9, Ayacucho, Peru. 
\vspace{0.5cm}

\item [\bf{8}.] Mamani-Cato RH, Huanca T, {\bf \underline{Mamani GC}} y Condori-Rojas N.
Modelacion de curvas de crecimiento de llamas Qara utilizando modelos de crecimiento no lineales
Reunion Cientifica Anual de la Asociacion Peruana de Produccion Animal. Agosto 7-9, Ayacucho, Peru. 
\vspace{0.5cm}

\item [\bf{7}.] {\bf \underline{Mamani GC}} y Ferraz JBS. 
Associacao entre coeficientes de endogamia genomicos e caracteristicas reprodutivas em bovinos Nelore.
II Simposio de Biociencia Animal. Setiembre 11, Pirassununga, SP, Brazil. 
\vspace{0.5cm}

\item [\bf{6}.] Quina E, Renieri C, Pena Y, {\bf \underline{Mamani GC}}.
Tendencias geneticas para peso de vellon, diametro y coeficiente de variabilidad de fibra de alpacas del Centro de Desarrollo Alpaquero Toccra
VII Congreso Mundial en Camelidos Sudamericanos. Octubre 28-30, Puno, Peru. 
\vspace{0.5cm}

\item [\bf{5}.] Mamani Torreblanca C, {\bf \underline{Mamani GC}}, Chavez J.
La llama carguera: cronicas contadas por um llamero del Peru
VII Congreso Mundial en Camelidos Sudamericanos. Octubre 28-30, Puno, Peru. 
\end{list1}
\vspace{0.5cm}

\section{\sc 2013}
\begin{list1}
\item [\bf{4}.] Gutierrez G, Candio J, Ruiz J, {\bf \underline{Mamani GC}}, Corredor A, Flores E.
Performance of alpacas from a dispersed open nucleus in Pasco region, Peru.
64 th EAAP Annual meeting. Symposium on South American Camelids and other Fibre Animals. August 25-30, Nantes, Francia. 
\vspace{0.5cm}

\item [\bf{3}.] {\bf \underline{Mamani GC}}, Huanca T, Gutierrez G.
Heredabilidad y tendencias geneticas para el peso al nacimiento en alpacas (\textit{Vicugna pacos}) del CIP Quimsachata-INIA, Puno.
XXXVI Reunion Cientifica Anual de la Asociacion Peruana de Produccion Animal. December 4-6, Lima, Peru. 
\end{list1}
\vspace{0.5cm}

\section{\sc 2012}
\begin{list1}
\item [\bf{2}.] Quina E, {\bf \underline{Mamani GC}}.
Indices reproductivos y caracterizacion fenotopica de llamas (\textit{Lama glama}) en Centro de Desarrollo Alpaquero Toccra - Yanque -Caylloma
VI Congreso Mundial en Camelidos Sudamericanos. Arica, Chile. 
\end{list1}
\vspace{0.5cm}

\section{\sc 2006}
\begin{list1}
\item [\bf{1}.] {\bf \underline{Mamani GC}}, Coila A.
Relacion de la morfologia, motilidad y vigor espermatico de semen de alpacas con la actividad enzimatica de las transaminasas GOT y GPT.
IV Congreso Mundial en Camelidos Sudamericanos. Noviembre, Santa Maria, Salta, Argentina. 
\end{list1}




\vspace{2cm}
\section{\sc Aditional Training} 
\vspace{2cm}

\section{\sc 2024}

\begin{list2}
\item Course: "Programming and computer algorithms in animal breeding with a focus on single-step GBLUP and genomic selection in practice". University of Georgia. May 13 to 31. Taught by Ignacy Misztal, Matias Bermann, Daniela Lourenco, Jorge Hidalgo.
\vspace{0.5cm}
\item Course: "An introduction to Bayesian Statistics with MCMC for Animal Geneticists". Universitat Politecnica de Valencia. April, 16 to 19. Taught by Agustin Blasco and Marina Martinez
\vspace{0.5cm}
\item Course: "ISB001: Breeding Programme Modelling with AlphaSimR". course of study offered by EdinburghX, an online learning initiative of University of Edinburg.
\end{list2}

\section{\sc 2023}

\begin{list2}
\item Course: "Curso Internacional Teorico Practico en Análisis Metagenómico". Chachapoyas - Perú.July 19 to 21.  
\end{list2}

\vspace{0.5cm}

\section{\sc 2021}

\begin{list2}
\item Course: "Manejo y aprovechamiento de la vicuña". Colegio del Ingenieros del Perú. Junín - Perú. 09 de julio al 09 de octubre. Dictado por reconocidos investigadores nacionales.
\vspace{0.5cm}
\item Course: "Metagenómica". Bioscience-App. Lima - Peru. 20 al 24 de setiembre. Taught by Luis Lozano Garcia.
\vspace{0.5cm}
\item Course: "Introducción a Data Science: Programación Estadística con R". Coursera \italic{on line}. 23 de marzo. 
\vspace{0.5cm}
\item Course: "III Curso Internacional de Biología de la Conservación". CONOPA - Instituto de Investigación y Desarrollo de Camélidos Sudamericanos. Lima - Peru. 22 - 26 de febrero. Taught by Jane Wheeler, Pablo Orozco-terWengel.
\end{list2}

\vspace{0.5cm}

\section{\sc 2020}
\begin{list2}

\item Course: "Estadísticas con R para la Caracterización Morfológica de Germoplasma". Universidad Nacional Agraria La Molina. Lima - Peru. 10 - 11 octubre. Taught by Carlos Arbizu.
\end{list2}
\vspace{0.5cm}



\section{\sc 2018}
\begin{list2}
\item Workshop: "Linear Mixed Model". University Federal of Vicosa. Minas Gerais-Brazil. October 26. Taught by Gota Morota and Alencar Xavier.
\vspace{0.5cm}
\item Short course: "Genome Sequencing for Genomic Studies". University of State of Sao Paulo. Jaboticabal-Brazil. May 18-22. Taught by Ben Hayes.
\vspace{0.5cm}
\item Course: "Controle do desenvolvimento folicular e da ovulacao". University of Sao Paulo. Butanta-SP-Brazil. Apr 16-27. Taught by Pietro Baruselli.
\vspace{0.5cm}
\end{list2}

\section{\sc 2017}
\begin{list2}
\item Course: "Population Genetics". University of Nebraska-Lincoln. Lincoln, USA. August-December. Taught by Jessica Petersen
\vspace{0.5cm}
\item Course: "Linear Models in Animal Breeding". University of Nebraska-Lincoln. Lincoln, USA. August-December. Taught by Matt Spangler
\vspace{0.5cm}
\item Course: "Statistical Method I". University of Nebraska-Lincoln. Lincoln, USA. August-December. Taught by Walter Stroup
\vspace{0.5cm}
\item Short course: "Introduction to Graphical Models with Applications to Quantitative Genetics and Genomics". Iowa State University. Ames-Iowa. June 19-23. Taught by Guillherme J.M. Rosa and Francisco Penagaricano
\vspace{0.5cm}
\end{list2}

\section{\sc 2016}
\begin{list2}
\item Short course: "Topicos Especiais sobre Crescimento e Desenvolvimento Animal". University of Sao Paulo. Pirassununga, Brazil. October 24 -30. Taught by David Gerrard
\vspace{0.5cm}
\item Course: "Modelagem Estatistico aplicado ao Melhoramento Animal". University of State of Sao Paulo. Jaboticabal-Brazil. November 21-25. Taught by Guillherme J.M. Rosa. 
\vspace{0.5cm}
\item Short course: "Iniciacao aos programas da familia BLUPF90". University of Sao Paulo. Pirassununga-Brazil. September 20 - 21. Taught by Mario Luiz Santana Junior.
\vspace{0.5cm}
\item Short course: "Quantitative Genetics and Genomics". University of Sao Paulo. Piracicaba- Brazil. Mayo 16 - 25. Taught by Gota Morota and Matt Spangler.
\vspace{0.5cm}
\item Course: "Genetica dos Processos Fisiologicos em Animais Domesticos ". University of Sao Paulo. Piracicaba-Brazil. March to  June. Taught by Luiz Coutinho and Gerson Mourao.
\vspace{0.5cm}
\item Course: "Analise de Dados Genomicos". University of Satate of Sao Paulo. Jaboticabal - Brazil. March to June. Taught by Roberto Carvalheiro.
\end{list2}  
\vspace{0.5cm}

\section{\sc 2015}

\begin{list2}
\item Short course: "Prediction of Complex Traits" (on-line). Universidad Politecnica de Valencia . Valencia-Spain. September 7-9. Taught by Daniel Gianola.
\vspace{0.5cm}
\item Course: "Principios de Associacao e Selecao Genomica". University of State of Sao Paulo. Jaboticabal-Brazil. August to December. Taught by Fernando Baldi.
\vspace{0.5cm}
\item Course: "Introducao a Inferencia Bayesiana aplicada ao Melhoramento Animal". University of State of Sao Paulo. Jaboticabal-Brazil. August to December. Taught by Henrique Nunes de Oliveira.
\vspace{0.5cm}
\item Short course: "Livestock Conservation Genomics: Data, Tools and Trends". University of State of Sao Paulo. Jaboticabal-Brazil. August 17-22. Taught by Ino Curik, Johann Solkner, Yuri Utsunomiya and Adam Utsunomiya
\vspace{0.5cm}
\end{list2}  
\vspace{0.5cm}

\section{\sc 2014}

\begin{list2}
\item Course: "Endocrinologia Molecular dos Processos Reprodutivos". University of Sao Paulo. Pirassununga-Brazil. September to October.
Taught by Mario Binelli.
\vspace{0.5cm}
\item Course: "Ferramentas para Avaliacao Espermatica: Sondas Fluorescentes e Sistema Computadorizado da Motilidade (CASA). University of Sao Paulo. Pirassununga-Brazil. October to November.
Taught by Rubens Paes de Arruda.
\end{list2}  
\vspace{0.5cm}

\section{\sc 2013}

\begin{list2}
\item Course: "Genetica Aplicada al Manejo de los Recursos Zootecnicos". University National Agrarian La Molina. Lima-Peru. December 9-14.
Taught by Abel Ponce de Leon.
\end{list2}  
\vspace{0.5cm}

\section{\sc 2012}

\begin{list2}
\item Course: "Biotecnologia Reproductiva en Animales de Granja". University National Agrarian La Molina. Lima-Peru. September to December.
Taught by Edwin Mellisho Salas and others.
\end{list2}  
\vspace{0.5cm}


\newpage

\section{\sc Teaching}

{\bf College of Animal Science. Toribio Rodriguez de Mendoza National University}, Amazonas, Perú.\\
% \vspace{-0.8cm}
\textbf{Associate Professor} \\
Course: Animal Phisiology    \hfill {\bf July to December - 2023}\\
Course: Animal Biometric    \hfill {\bf July to December  2023}\\

{\bf Master Program in Animal Science. Toribio Rodriguez de Mendoza National University}, Amazonas, Perú.\\
% \vspace{-0.8cm}
\textbf{Associate Professor} \\
Course: Animal Breeding    \hfill {\bf November - 2023}\\
Course: Animal Genetic    \hfill {\bf December  2023}\\


{\bf PhD Program in Animal Science. Altiplano National University}, Puno, Perú.\\
% \vspace{-0.8cm}
\textbf{Visitor Professor} \\

Course: Animal Genetic   \hfill {\bf November to December 2023}\\
Course: Animal Genetic    \hfill {\bf September to October 2022}\\
Course: Animal Genetic    \hfill {\bf August to September  2021}\\
Course: Animal Genetic    \hfill {\bf February to March  2021}\\
Course: Animal Genetic    \hfill {\bf December 2019 to January - 2020}\\

{\bf Master Program in Animal Science, EPG. Altiplano National University}, Puno, Perú.\\
% \vspace{-0.8cm}
\textbf{Visitor Professor} \\

Course: Animal Genomics  \hfill {\bf June - 2024}\\
Course: Animal Genomics  \hfill {\bf June - 2023}\\
Course: Animal Genomics  \hfill {\bf June - 2022}\\
Course: Animal Genomics  \hfill {\bf October - 2021}\\
Course: Quantitative and Populations Genetic   \hfill {\bf June  to July - 2021}\\
Course: Quantitative and Populations Genetic  \hfill {\bf December - 2020 to January - 2021}\\


{\bf Especialist Program in Camelids South Americans. Altiplano National University}, Puno, Perú.\\
% \vspace{-0.8cm}
\textbf{Visitor Professor} \\

Course: Animal Breeding and Prediction Breeding Value    \hfill {\bf Nov to December - 2023}\\
Course: Animal Breeding and Prediction Breeding Value    \hfill {\bf Nov to December - 2021}\\
Course: Animal Breeding and Prediction Breeding Value     \hfill {\bf March to April - 2021 }\\


{\bf College of Animal Science, University National Agraria La Molina}, Lima, Perú.\\
% \vspace{-0.8cm}
\textbf{Speaker} \\
Course: Aplicacíon de Herramientas Estadísticas y Computacionales para el Análisis de Datos Genómicos de Animales de Granja    \hfill {\bf 2-6 Abril, 2019}\\
\vspace{0.5cm}

{\bf College of Animal Science, University of Sao Paulo}, SP, Brazil\\
% \vspace{-0.8cm}
\textbf{Pedagogical preparation(Estagiario)} \\
Course: Estatistic II     \hfill {\bf Semestre 2017-I}\\
Course: Estatistic I      \hfill {\bf Semestre 2016-I}\\
\vspace{0.5cm}

{\bf College of Animal Science, University of Sao Paulo}, SP, Brazil\\
% \vspace{-0.8cm}
\textbf{Speaker} \\
Course: Genética Básica e Evolução (Carrera Zootecnia)   \hfill {\bf II Semestre, 2016}\\
Course: Genética Básica e Biologia Molecular (Carrera Medicina Veterinaria)   \hfill {\bf II Semestre, 2016}\\
\vspace{0.5cm}

{\bf College of Animal Science, University National Agraria La Molina}, Lima, Peru.\\

\vspace{-0.8cm}
%\vspace{-0.4cm}
\textbf{Practical assistant} \\
Course: Genética de Poblaciones en Mej. Animal (Maestria)     \hfill {\bf Semestre 2013-II} \\
Course: Mejoramiento Ganadero Avanzado (Carrera de Zootecnia)    \hfill {\bf Semestre 2013-II} \\


{\bf Instituto Nacional de Innovación Agraria}, Donoso-Huaral, Lima, Peru.\\

\vspace{-0.8cm}
%\vspace{-0.4cm}
\textbf{Speaker} \\
Course: Formación de PAT en Inseminación Artificial en ganado bovino     \hfill {\bf Cañete, Mayo, 2022} \\
Course: Formación de PAT en Inseminación Artificial en ganado bovino     \hfill {\bf Oxapampa, Setiembre, 2022} \\
Course: Formación de PAT en Inseminación Artificial en ganado bovino     \hfill {\bf Tingua-Ancash, Julio, 2022} \\
Course Virtual: Introducción al Lenguaje de Programación R     \hfill {\bf Donoso, Junio, 2022} \\
Course Virtual: Uso de la Genómica y los Registros Productivos     \hfill {\bf Donoso, Setiembre, 2021} \\
Course: Biotecnologías Reproductivas, IA, TE en vacunos     \hfill {\bf Donoso-Huaral, Febrero, 2022} \\


\newpage

\section{\sc Computer Skills} 
\begin{list2}
\item Programing languages: Python, Fortran
\vspace{0.3cm}
\item Statistical Computational Tools: R, Asreml, SAS, Plink, Beagle, FImpute and QGis
\vspace{0.3cm}
\item Simulation programs: Vensim, QMsim, AlphaSimR (R) and MOBPs (R)
\vspace{0.3cm}
\item Content-description Languages: \LaTeX
\vspace{0.3cm}
\item Operting systems: Mac Os X, Windows and Linux  
\end{list2}

\section{\sc Languages} 
\begin{list2}
\vspace{0.3cm}
\item Spanish
\vspace{0.3cm}
\item Quechua
\vspace{0.3cm}
\item Portuguese
\vspace{0.3cm}
\item English
\end{list2}

\section{\sc References}
References available upon request. 

\end{resume}
\end{document}
