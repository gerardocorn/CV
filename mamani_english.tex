\documentclass[margin,line,10pt]{res}

\usepackage{verbatim}
\usepackage[brazilian]{babel}
\usepackage[latin1]{inputenc}
\usepackage[portuguese]{babel}
\usepackage{hyperref}
\usepackage{color}
\usepackage{multicol}

\oddsidemargin -.5in
\evensidemargin -.5in
\textwidth=6.0in
\itemsep=0in
\parsep=0in

\newenvironment{list1}{
  \begin{list}{\ding{113}}{%
      \setlength{\itemsep}{0in}
      \setlength{\parsep}{0in} \setlength{\parskip}{0in}
      \setlength{\topsep}{0in} \setlength{\partopsep}{0in} 
      \setlength{\leftmargin}{0.17in}}}{\end{list}}

\newenvironment{list2}{
  \begin{list}{$\bullet$}{%
      \setlength{\itemsep}{0in}
      \setlength{\parsep}{0in} \setlength{\parskip}{0in}
      \setlength{\topsep}{0in} \setlength{\partopsep}{0in} 
      \setlength{\leftmargin}{0.2in}}}{\end{list}}

\begin{document}

\name{Gerardo Mamani \hspace{12.0cm} October 2017 \vspace*{.1in}}

\begin{resume}
\section{\sc Contact Information}
\vspace{.05in}
\begin{tabular}{@{}p{3in}p{4in}}
Deparment of Animal Science    & \hspace{2.5cm} {\it E-mail:}  gerardo.mamani@usp.br\\       
University of Nebraska-Lincoln  & \hspace{2.5cm} {\it Celular:} 958745068(Peru)\\     

\end{tabular}

\vspace{0.3cm}
\section{\sc Research Interestes}
I come from livestock family in the region of Cusco in Peru. I worked in rural development projects in Peru, with an emphasis in alpacas, llamas and guinea pigs systems production. Actually, I am a PhD student in Animal Bioscience of the University of Sao Paulo-Brazil. My interest is Genetics and Animal Breeding, Animal Reproduction and Statistics. 
 
\section{\sc Education}

{\bf University of Nebraska - Lincoln}, Lincoln, NE USA\\
\vspace*{-.1in}
\begin{list1}
\item[] Visitor Scholar, 4/2017 - actually
\begin{list2}
\vspace*{.05in}
\item Project: "Genomic prediction in Nellore cattle" 
\item Supervisor: Prof. Dr. Gota Morota
\end{list2}
\vspace*{.05in}
\end{list1}

{\bf University of Sao Paulo}, Pirassununga, Sao Paulo, Brazil\\
\vspace*{-.1in}
\begin{list1}
\item[] Ph.D. student, Animal Bioscience, September 2014 - actually
\begin{list2}
\vspace*{.05in}
\item Project of Thesis: "Association between coefficients of genomic inbreeding and productive and reproductive traits in Nellore cattle" 
\item Advisor: Prof. Dr. Jose Bento Sterman Ferraz 
\end{list2}
\vspace*{.05in}
\end{list1}

{\bf University National Agrarian La Molina}, Lima, Peru\\
\vspace*{-.1in}
\begin{list1}
\item[] Magister Scientiae, Animal Production, March 2013
\begin{list2}
\vspace*{.05in}
\item Thesis: "Genetic structure and genetic trend of birth weight in alpacas of the Germoplasm Bank of Quimsachata Puno - Peru". 
\item Advisor: Prof. Dr. Gustavo Gutierrez Reynoso 
\item Committee: Drs. Juan Chavez Cossio, Jorge Aliaga and Edwin Mellisho Salas
\end{list2}
\vspace*{.05in}
\end{list1}

{\bf University National of the Altiplano}, Puno, Peru\\
\vspace*{-.1in}
\begin{list1}
\item[] Veterinary Doctor,  Marzo 2005
\begin{list2}
\vspace*{.05in}
\item Tesis: "Relationship between sperm morphology, motility and vigor and the enzymatic activity of GOT and GPT transaminases in alpacas". 
\item Advisor: Prof. Dr. Pedro Coila 
\item Committee: Drs. Maximo Melo, Guido Perez and Zenon Maquera
\end{list2}
\end{list1}

\section{\sc Work \phantom{1cm} Experience}

Project: Improving Alpaca  Systems Production in Highland of the Central Sierra of Peru\\
{\bf University National of La Molina - VLIR}, Lima - Pasco, Peru\\
\vspace{-.3cm}
\textbf{Cargo:} Research assistant \hfill {\bf 08/2013 - 08/2014}\\

Elaboration of baseline and technical expedient of livestock development in Orcopampa, Arequipa\\
{\bf IPTIG}, Lima, Peru\\
\vspace{-.3cm}
\textbf{Cargo:} Consultant  \hfill {\bf 05/2013 - 06/2013}\\

Project: Improvement of the capacity and employment for the alpaca production in the Northeast of Puno, Peru\\
{\bf CECOALP}, Puno, Peru\\
\vspace{-.3cm}
\textbf{Cargo:} Extension.  \hfill {\bf 04/2010 - 08/2010}\\

Study of baseline and elaboration of the project for the development of South American domestic camelids in Antauta and Ajoyani, Puno.\\
{\bf ONG DESCO}, Puno, Peru\\
\vspace{-.3cm}
\textbf{Cargo:} Consultant.  \hfill {\bf 03/2010}\\

Project: Development of productive and commercial capacities of the small breeders of alpacas of Manazo and Cabanillas, Puno.\\
{\bf ONG CEDER}, Puno, Peru\\
\vspace{-.3cm}
\textbf{Cargo:} Extension.  \hfill {\bf 01/2008 - 12/2009}\\

Project: Desarrollo de capacidades productivas y comerciales de los pequenos criadores de alpacas de las comunidades altoandinas de Moquegua y Arequipa.\\
{\bf ONG CEDER - CESVI}, Puno, Peru\\
\vspace{-.3cm}
\textbf{Cargo:} Extension.  \hfill {\bf 11/2006 - 12/2007}\\

Programa de Apoyo a Campesinos Pastores de Alturas PROALPACA-Huancavelica.\\
{\bf ONG Vecinos Peru}, Ayacucho-Huancavelica, Peru\\
\vspace{-.3cm}
\textbf{Cargo:} Extension.  \hfill {\bf 11/2005 - 10/2006}\\

Elaboracion de linea de base del Project de desarrollo de capacidades, incremento productivo y mercadeo de leche y derivados de las comunidades campesinas del altiplano de Puno CARITAS- Puno.\\
{\bf ONG DESCO}, Puno, Peru\\
\vspace{-.3cm}
\textbf{Cargo:} Encuestador.  \hfill {\bf 10/2005}\\

Inversiones Doria SRLtda. Santa Lucia - Puno.\\
{\bf Project de Desarrollo Corredor Puno -Cusco}, Puno, Peru\\
\vspace{-.3cm}
\textbf{Cargo:} Extension.  \hfill {\bf 10/2005}\\

\vspace{0.5cm}

\section{\sc Preprints}

\begin{list1}

\item [{\bf 3}.]  {\bf \underline{Mamani GC}}, Mendoza JG, Gutierrez G.   
     Caracterizacion de poblaciones de llamas mediantes analise multivariado. In prep. %To Appear. 
\vspace{0.5cm}
     
\item [{\bf 2}.]  {\bf \underline{Mamani GC}}, Santana B, Perez B, Oliveira G, Mattos E, Ferraz JB.   
     Diversidad genetica de ovejas Santa Ines usando datos de pedigre e genomicos. In prep. %To Appear. 
\vspace{0.5cm}

\item [{\bf 1}.]  {\bf \underline{Mamani GC}}, Mamani-Cato RH, Huanca T, Gutierrez G, Gutierrez JP.   
     Estructura genetica de alpacas del Banco de Germoplasma del INIA Peru. In prep. %To Appear. 

\end{list1}
\vspace{0.5cm}

\vspace{0.5cm}
\section{\sc Journal Papers}

\vspace{0.9cm}

\section{\sc 2017}

\begin{list1}
\item [{\bf 3}.]  Santana B, {\bf \underline{Mamani GC}}, Oliveira G, Castro L, Molento M, Ventura R, and Ferraz JBS .  
     Characterization of runs of homozygosity in a population of Thoroughbred Horses.   
     \textcolor{black}{\href{http://www.citec.fatecjab.edu.br/index.php/files/article/view/1161}{{\it Revista Ciencia & Tecnologia}.{\bf v9}.2017}. } 
     \vspace{0.5cm}
     \item [{\bf 2}.] Santana B, Fonseca R, Matos M, {\bf \underline{Mamani GC}}, Eler J and Ferraz JBS.  
     Feasibility of using "days to a specific weight" traits in Nellore cattle breeding programs.   
     \textcolor{black}{\href{http://www.scielo.br/scielo.php?script=sci_arttext&pid=S1519-99402017000200260}{{\it Revista Brasileira de Saude e Producao Animal}. {\bf 18}.2}}
\end{list1}
\vspace{0.5cm}

\section{\sc 2015}

\begin{list1}
\item [{\bf 1}.]  Vargas A, Gutierrez G. and {\bf \underline{Mamani GC}}.  
 An application of Gibbs sampling for genetic parameters estimation in guinea pigs using MCMCglmm.   
     \textcolor{black}{\href{http://dev.scielo.org.pe/scielo.php?script=sci_arttext&pid=S1609-91172015000200003&lng=en&nrm=iso}{{\it Revista de Investigaciones Veterinarias del Peru}. {\bf 26}.2 } } 
\end{list1}
\vspace{0.5cm}

\vspace{0.5cm}
\section{\sc Proceedings abstract}
\vspace{0.9cm}

\section{\sc 2017}
\begin{list1}
\item [\bf{12}.] {\bf \underline{Mamani GC}}, Santana B,  Ventura R, Ferraz JBS. 
Caracterizacao de segmentos de homozigose en bovinos, ovinos, porcinos e equinos.
VI Simposio de Biociencia Animal . September 5, Pirassununga, SP, Brazil. 
\vspace{0.5cm}
\item [\bf{11}.] Santana B, {\bf \underline{Mamani GC}}, Oliveira G,  Ventura R, Ferraz JBS. 
Efeitos da endogamia sobre a uniformidade de progenie de touros nelore..
VI Simposio de Biociencia Animal. September 5, Pirassununga, SP, Brazil. 
\end{list1}

\section{\sc 2016}
\begin{list1}
\item [\bf{10}.] {\bf \underline{Mamani GC}}, and Santana B, Aguirre L, Mattos EC, Ferraz JBS. 
Estructura genetica de ovinos Santa Ines por analise de pedigree.
III Simposio de Biociencia Animal. Setiembre 8, Pirassununga, SP, Brazil. 
\end{list1}

\section{\sc 2015}
\begin{list1}
\item [\bf{9}.] Mamani-Cato RH, Huanca T, Condori-Rojas N, {\bf \underline{Mamani GC}}.
Estructura genetica de la poblacion de llamas del Banco de Germoplasma del INIA Peru
Reunion Cientifica Anual de la Asociacion Peruana de Produccion Animal. Agosto 7-9, Ayacucho, Peru. 
\vspace{0.5cm}

\item [\bf{8}.] Mamani-Cato RH, Huanca T, {\bf \underline{Mamani GC}} y Condori-Rojas N.
Modelacion de curvas de crecimiento de llamas Qara utilizando modelos de crecimiento no lineales
Reunion Cientifica Anual de la Asociacion Peruana de Produccion Animal. Agosto 7-9, Ayacucho, Peru. 
\vspace{0.5cm}

\item [\bf{7}.] {\bf \underline{Mamani GC}} y Ferraz JBS. 
Associacao entre coeficientes de endogamia genomicos e caracteristicas reprodutivas em bovinos Nelore.
II Simposio de Biociencia Animal . Setiembre 11, Pirassununga, SP, Brazil. 
\vspace{0.5cm}

\item [\bf{6}.] Quina E, Renieri C, Pena Y, {\bf \underline{Mamani GC}}.
Tendencias geneticas para peso de vellon, diametro y coeficiente de variabilidad de fibra de alpacas del Centro de Desarrollo Alpaquero Toccra
VII Congreso Mundial en Camelidos Sudamericanos. Octubre 28-30, Puno, Peru. 
\vspace{0.5cm}

\item [\bf{5}.] Mamani Torreblanca C, {\bf \underline{Mamani GC}}, Chavez J.
La llama carguera: cronicas contadas por um llamero del Peru
VII Congreso Mundial en Camelidos Sudamericanos. Octubre 28-30, Puno, Peru. 
\end{list1}
\vspace{0.5cm}

\section{\sc 2013}
\begin{list1}
\item [\bf{4}.] Gutierrez G, Candio J, Ruiz J, {\bf \underline{Mamani GC}}, Corredor A, Flores E.
Performance of alpacas from a dispersed open nucleus in Pasco region, Peru.
64 th EAAP Annual meeting. Symposium on South American Camelids and other Fibre Animals. August 25-30, Nantes, Francia. 
\vspace{0.5cm}

\item [\bf{3}.] {\bf \underline{Mamani GC}}, Huanca T, Gutierrez G.
Heredabilidad y tendencias geneticas para el peso al nacimiento en alpacas (\textit{Vicugna pacos}) del CIP Quimsachata-INIA, Puno.
XXXVI Reunion Cientifica Anual de la Asociacion Peruana de Produccion Animal. Diciembre 4-6, Lima, Peru. 
\end{list1}
\vspace{0.5cm}

\section{\sc 2012}
\begin{list1}
\item [\bf{2}.] Quina E, {\bf \underline{Mamani GC}}.
Indices reproductivos y caracterizacion fenotopica de llamas (\textit{Lama glama}) en Centro de Desarrollo Alpaquero Toccra - Yanque -Caylloma
VI Congreso Mundial en Camelidos Sudamericanos. Arica, Chile. 
\end{list1}
\vspace{0.5cm}

\section{\sc 2006}
\begin{list1}
\item [\bf{1}.] {\bf \underline{Mamani GC}}, Coila A.
Relacion de la morfologia, motilidad y vigor espermatico de semen de alpacas con la actividad enzimatica de las transaminasas GOT y GPT.
IV Congreso Mundial en Camelidos Sudamericanos. Noviembre, Santa Maria, Salta, Argentina. 
\end{list1}

\vspace{0.9cm}
\section{\sc Aditional Training} 
\vspace{2cm}

\section{\sc 2017}
\begin{list2}
\item Course: "Population Genetics". University of Nebraska-Lincoln. Lincoln, USA. August-December. Taught by Jessica Petersen
\vspace{0.5cm}
\item Course: "Linear Models in Animal Breeding". University of Nebraska-Lincoln. Lincoln, USA. August-December. Taught by Matt Spangler
\vspace{0.5cm}
\item Course: "Statistical Method I". University of Nebraska-Lincoln. Lincoln, USA. August-December. Taught by Walter Stroup
\vspace{0.5cm}
\item Short course: "Introduction to Graphical Models with Applications to Quantitative Genetics and Genomics". Iowa State University. Ames-Iowa. June 19-23. Taught by Guillherme J.M. Rosa and Francisco Penagaricano
\vspace{0.5cm}
\end{list2}  

\section{\sc 2016}

\begin{list2}
  \item Short course: "Topicos Especiais sobre Crescimento e Desenvolvimento Animal". University of Sao Paulo. Pirassununga, Brazil. October 24 -30. Taught by David Gerrard
  \vspace{0.5cm}
  \item Course: "Modelagem Estatistico aplicado ao Melhoramento Animal". University of State of Sao Paulo. Jaboticabal-Brazil. November 21-25. Taught by Guillherme J.M. Rosa. 
  \vspace{0.5cm}
  \item Short course: "Iniciacao aos programas da familia BLUPF90". University of Sao Paulo. Pirassununga-Brazil. September 20 - 21. 
  Taught by Mario Luiz Santana Junior.
  \vspace{0.5cm}
  \item Short course: "Quantitative Genetics and Genomics". University of Sao Paulo. Piracicaba- Brazil. Mayo 16 - 25.
  Taught by Gota Morota and Matt Spangler.
  \vspace{0.5cm}
    \item Course: "Genetica dos Processos Fisiologicos em Animais Domesticos ". University of Sao Paulo. Piracicaba-Brazil. March to  June. Taught by Luiz Coutinho and Gerson Mourao.
  \vspace{0.5cm}
   \item Course: "Analise de Dados Genomicos". University of Satate of Sao Paulo. Jaboticabal - Brazil. March to June.
  Taught by Roberto Carvalheiro.
\end{list2}  
\vspace{0.5cm}

\section{\sc 2015}

\begin{list2}
  \item Short course: "Prediction of Complex Traits" (on-line). Universidad Politecnica de Valencia . Valencia-Spain. September 7-9. Taught by Daniel Gianola.
  \vspace{0.5cm}
  \item Course: "Principios de Associacao e Selecao Genomica". University of State of Sao Paulo. Jaboticabal-Brazil. August to December. Taught by Fernando Baldi.
  \vspace{0.5cm}
  \item Course: "Introducao a Inferencia Bayesiana aplicada ao Melhoramento Animal". University of State of Sao Paulo. Jaboticabal-Brazil. August to December. Taught by Henrique Nunes de Oliveira.
  \vspace{0.5cm}

  \item Short course: "Livestock Conservation Genomics: Data, Tools and Trends". University of State of Sao Paulo. Jaboticabal-Brazil. August 17-22. Taught by Ino Curik, Johann Solkner, Yuri Utsunomiya and Adam Utsunomiya
  \vspace{0.5cm}
\end{list2}  
\vspace{0.5cm}

\section{\sc 2014}

\begin{list2}
\item Course: "Endocrinologia Molecular dos Processos Reprodutivos". University of Sao Paulo. Pirassununga-Brazil. September to October.
Taught by Mario Binelli.
\vspace{0.5cm}
\item Course: "Ferramentas para Avaliacao Espermatica: Sondas Fluorescentes e Sistema Computadorizado da Motilidade (CASA). University of Sao Paulo. Pirassununga-Brazil. October to November.
Taught by Rubens Paes de Arruda.
\end{list2}  
\vspace{0.5cm}

\section{\sc 2013}

\begin{list2}
\item Course: "Genetica Aplicada al Manejo de los Recursos Zootecnicos". University National Agrarian La Molina. Lima-Peru. December 9-14.
Taught by Abel Ponce de Leon.
\end{list2}  
\vspace{0.5cm}

\section{\sc 2012}

\begin{list2}
\item Course: "Biotecnologia Reproductiva en Animales de Granja". University National Agrarian La Molina. Lima-Peru. September to December.
Taught by Edwin Mellisho Salas and others.
\end{list2}  
\vspace{0.5cm}

\vspace{0.5cm}
\section{\sc Teaching}

{\bf Facultade de Zootecnia e Ingenieria de Alimentos, Universidade de Sao Paulo}, SP, Brasil\\

\vspace{.01pt}
\textbf{Speaker}\\
Curse: Genetic Basic     \hfill {\bf Semester 2016-II} \\

\vspace{.01pt}
\textbf{Pedagogical preparation} \\
Curse: Estatistics II     \hfill {\bf Semester 2016-I}\\
\vspace{0.5cm}


{\bf Facultad de Zootecnia, Universidad Nacional Agraria La Molina}, Lima, Peru.\\
\vspace{.01pt}

\textbf{Assistant of practics} \\
Curse: Population genetics     \hfill {\bf Semester 2013-II} \\
\vspace{0.5cm}


\section{\sc Computer Skills} 
\begin{list2}
\item Programing languages: Python
\vspace{0.3cm}
\item Statistical Computational Tools : R, SAS, Asreml, Plink and Blupf90
\vspace{0.3cm}
\item Simulation programs: Vensim, QMsim
\vspace{0.3cm}
\item Content-description Languages: \LaTeX
\vspace{0.3cm}
\item Operting systems : Windows, Linux and Mac Os X 
\end{list2}

\section{\sc Languages} 
\begin{list2}
\vspace{0.3cm}
\item Spanish
\vspace{0.3cm}
\item Quechua
\vspace{0.3cm}
\item Portuguese
\vspace{0.3cm}
\item English
\end{list2}

\section{\sc References}
References available upon request. 

\end{resume}
\end{document}
