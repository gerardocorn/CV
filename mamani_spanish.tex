\documentclass[margin,line,10pt]{res}

\usepackage{verbatim}
\usepackage[brazilian]{babel}
\usepackage[spanish]{babel}
%\usepackage[latin1]{inputenc}
\usepackage[utf8]{inputenc}
\usepackage{hyperref}
\usepackage{color}
\usepackage{multicol}

\oddsidemargin -.5in
\evensidemargin -.5in
\textwidth=6.0in
\itemsep=0in
\parsep=0in

\newenvironment{list1}{
  \begin{list}{\ding{113}}{%
      \setlength{\itemsep}{0in}
      \setlength{\parsep}{0in} \setlength{\parskip}{0in}
      \setlength{\topsep}{0in} \setlength{\partopsep}{0in} 
      \setlength{\leftmargin}{0.17in}}}{\end{list}}
\newenvironment{list2}{
  \begin{list}{$\bullet$}{%
      \setlength{\itemsep}{0in}
      \setlength{\parsep}{0in} \setlength{\parskip}{0in}
      \setlength{\topsep}{0in} \setlength{\partopsep}{0in} 
      \setlength{\leftmargin}{0.2in}}}{\end{list}}

\begin{document}

\name{Gerardo C. Mamani\hspace{10.5cm} Febrero 2018 \vspace*{.1in}}

\begin{resume}
\section{\sc Contacto}
\vspace{.05in}
\begin{tabular}{@{}p{3in}p{4in}}
Facultad de Zootecnia e Ingenieria de Alimentos,    & \hspace{2.3cm} {\it E-mail:}  gerardo.mamani@usp.br\\       
Pirassununga - São Paulo  & \hspace{2.3cm} {\it Celular:} 19 971589025 (Brasil)\\     

\end{tabular}

\vspace{0.3cm}
\section{\sc Reseña e interés}
Trabajé en proyectos de desarrollo rural en Huancavelica, Moquegua, Arequipa, Puno, Pasco y Junín; con énfasis en producción de alpacas, llamas y cuyes. Actualmente soy estudiante de doctorado en Biociencia Animal en la Universidad de São Paulo. Con áreas de interés en Genética y Mejoramiento Animal y Reproducción Animal. Fui revisor de la Revista de la Asociación Peruana de Reproducción Animal SPERMOVA
%In particular, I am working on integrating the different types of omics data for the prediction of complex traits using a broad range of agricultural species. 
 
\section{\sc Estudios}

{\bf University of Nebraska - Lincoln}, Lincoln, NE, USA\\
\vspace*{-.1in}
\begin{list1}
\item[] Visitor Scholar, 4/2017 - 12/2017
\begin{list2}
\vspace*{.05in}
\item Proyecto: "Genomic prediction in Nellore cattle" 
\item Supervisor: Prof. Dr. Gota Morota
\end{list2}
\vspace*{.05in}
\end{list1}

{\bf Universidade de São Paulo}, Pirassununga, São Paulo - Brasil\\
\vspace*{-.1in}
\begin{list1}
\item[] Estudiante de Doctorado, Biociencia Animal, 2014-Actual
\begin{list2}
\vspace*{.05in}
\item Proyecto de Tesis: "Asociación entre coeficientes de endogamia genômicos y caracteristicas productivas y reproductivas en ganado Nelore." 
\item Orientador: Prof. Dr. José Bento Sterman Ferraz 
\end{list2}
\vspace*{.05in}
\end{list1}

{\bf Universidad Nacional Agraria La Molina}, Lima, Perú\\
\vspace*{-.1in}
\begin{list1}
\item[] Magister Scientiae, Producción Animal, Marzo 2013
\begin{list2}
\vspace*{.05in}
\item Tesis: ``Estructura genética poblacional y tendencia genética de peso vivo al nacimiento en alpacas del Banco de Germoplasma de Quimsachata del INIA en Puno." 
\item Asesor: Prof. Dr. Gustavo Gutiérrez Reynoso 
\item Jurado de Tesis: Drs. Juan Chávez Cossio, Jorge Aliaga y Edwin Mellisho Salas
\end{list2}
\vspace*{.05in}
\end{list1}


{\bf Universidad Nacional del Altiplano}, Puno, Perú\\
\vspace*{-.1in}
\begin{list1}
\item[] Médico Veterinario y Zootecnista,  Marzo 2005
\begin{list2}
\vspace*{.05in}
\item Tesis:  "Relación de la morfología, motilidad y vigor espermático de semen de alpacas con la actividad enzimática de las transaminasas GOT y GPT." 
\item Asesor: Prof. Dr. Pedro Coila 
\item Jurado Tesis: Drs. Máximo Melo, Guido Pérez y Zenón Maquera
\end{list2}
\end{list1}

\section{\sc Experiencia \phantom{1cm} Laboral}

Proyecto: Mejorando Sistemas de Producción de Alpacas en Pastizales de Sierra Central del Perú\\
{\bf Universidad Nacional Agraria La Molina - VLIR}, Lima - Pasco, Perú\\
\vspace{-.3cm}
\textbf{Cargo:} Asistente de investigación \hfill {\bf 08/2013 - 08/2014}\\
 
Elaboración de línea de base y de expediente técnico de desarrollo pecuario en Anexo Lontojoya -Orcopampa - Arequipa\\
{\bf Instituto Peruano de Tecnología, Innovación y Gestión - IPTIG}, Lima-Arequipa, Perú\\
\vspace{-.3cm}
\textbf{Cargo:} Consultor  \hfill {\bf 05/2013 - 06/2013}\\

Proyecto: Desarrollo de capacidades y generación de empleo para la producción alpaquera en la región Noreste de Puno\\
{\bf Central de Cooperativas de Servicios Alpaqueras CECOALP}, Puno, Perú\\
\vspace{-.3cm}
\textbf{Cargo:} Residente Especialista en Camélidos Sudamericanos.  \hfill {\bf 04/2010 - 08/2010}\\

Estudio de línea de base y elaboración del Proyecto de desarrollo de los camélidos sudamericanos domésticos en el distrito de Antauta (Melgar), Ajoyani (Carabaya) y zonas de influencia.\\
{\bf Centro de Estudios y Promoción del Desarrollo DESCO}, Puno, Perú\\
\vspace{-.3cm}
\textbf{Cargo:} Consultor.  \hfill {\bf 03/2010}\\

Proyecto: Desarrollo de capacidades productivas y comerciales de los pequeños criadores de alpacas de los Distritos de Mañazo y Cabanillas del Departamento de Puno.\\
{\bf Centro de Estudios para el Desarrollo Regional CEDER}, Puno, Perú\\
\vspace{-.3cm}
\textbf{Cargo:} Especialista en Comercialización de Fibra de Alpaca.  \hfill {\bf 01/2008 - 12/2009}\\

Proyecto: Desarrollo de capacidades productivas y comerciales de los pequeños criadores de alpacas de las comunidades altoandinas de Moquegua y Arequipa.\\
{\bf Centro de Estudios para el Desarrollo Regional CEDER - CESVI}, Puno, Perú\\
\vspace{-.3cm}
\textbf{Cargo:} Responsable de Campo.  \hfill {\bf 11/2006 - 12/2007}\\

Programa de Apoyo a Campesinos Pastores de Alturas PROALPACA-Huancavelica.\\
{\bf ONG Vecinos Perú}, Ayacucho-Huancavelica, Perú\\
\vspace{-.3cm}
\textbf{Cargo:} Responsable de Campo.  \hfill {\bf 11/2005 - 10/2006}\\

Elaboración de línea de base del proyecto de desarrollo de capacidades, incremento productivo y mercadeo de leche y derivados de las comunidades campesinas del altiplano de Puno CARITAS- Puno.\\
{\bf Centro de Estudios y Promoción del Desarrollo DESCO}, Puno, Perú\\
\vspace{-.3cm}
\textbf{Cargo:} Encuestador.  \hfill {\bf 10/2005}\\

Oferente Técnico en Crianza de Alpacas - Inversiones Doria S.R. Ltda. Santa Lucia - Puno.\\
{\bf Proyecto de Desarrollo Corredor Puno -Cusco}, Puno, Perú\\
\vspace{-.3cm}
\textbf{Cargo:} Oferente técnico.  \hfill {\bf 10/2005}\\

\vspace{0.5cm}

\section{\sc Articulo en Preparación}

\begin{list1}

\item [{\bf 3}.]  {\bf \underline{Mamani GC}}, Mendoza JG,Gutierrez G.   
     Caracterización de poblaciones de llamas mediantes analise multivariado. In prep. %To Appear. 
\vspace{0.5cm}
     
\item [{\bf 2}.]  {\bf \underline{Mamani GC}}, Santana B, Perúz B, Oliveira G, Mattos E, Ferraz JB.   
     Diversidad genética de ovejas Santa Ines usando datos de pedigré e genômicos. In prep. %To Appear. 
\vspace{0.5cm}

\item [{\bf 1}.]  {\bf \underline{Mamani GC}}, Mamani-Cato RH, Huanca T, Gutierrez G, Gutierrez JP.   
     Estructura genética de alpacas del Banco de Germoplasma del INIA Perú. In prep. %To Appear. 

\end{list1}
\vspace{0.5cm}

\vspace{0.5cm}
\section{\sc Artículo publicado}

\vspace{0.9cm}
\section{\sc 2015}

\begin{list1}
\item [{\bf 1}.]  Vargas A, Gutiérrez G. and {\bf \underline{Mamani GC}}.  
 An application of Gibbs sampling for genetic parameters estimation in guinea pigs using MCMCglmm.   
     \textcolor{blue}{\href{http://dev.scielo.org.pe/scielo.php?script=sci_arttext&pid=S1609-91172015000200003&lng=en&nrm=iso}{ {\it Revista de Investigaciones Veterinarias del Perú}. {\bf 5}:56 }. } 
     \end{list1}
\vspace{0.5cm}

\vspace{0.5cm}
\section{\sc Resúmenes en Congresos}
\vspace{0.9cm}

\section{\sc 2018}
\begin{list1}

\item [\bf{16}.] {\bf \underline{Mamani GC}}, Santana B, Mattos E, Eler J, Morota G, Ferraz JBS. Effect of genomic inbreeding in productive traits in Santa Ines sheep.
XXVI Reunion de la Asociacion Latinoamericana de Produccion Animal. Guayaquil, Ecuador. 
\vspace{0.5cm}
\item [\bf{15}.] Santana B, {\bf \underline{Mamani GC}}, Mattos E, Eler J, Ferraz JBS. 
Estrutura populacional em racas bovinas usando dados genomicos. 
XXVI Reunion de la Asociacion Latinoamericana de Produccion Animal. Guayaquil, Ecuador. 
\vspace{0.5cm}
\item [\bf{14}.] {\bf \underline{Mamani GC}}, Santana B, Oliveira G, Ventura R, Mattos E, Eler J, Morota G, Ferraz JBS. 
Effect of inbreeding in productive traits in Nellore cattle.
XI World Congress on Genetics Applied to Livestock Production. Auckland, New Zeland. 
\vspace{0.5cm}
\item [\bf{13}.] Santana B, {\bf \underline{Mamani GC}}, Oliveira G, Ventura R, Mattos E, Eler J, Ferraz JBS. 
Association of levels of homozygosity in Nellore bulls with high and low variability of estimated breeding values of their progenies.
XI World Congress on Genetics Applied to Livestock Production. Auckland, New Zeland. 
\end{list1}

\section{\sc 2017}
\begin{list1}
\item [\bf{12}.] {\bf \underline{Mamani GC}}, Santana B,  Ventura R, Ferraz JBS. 
Caracterizacao de segmentos de homozigose en bovinos, ovinos, porcinos e equinos.
VI Simposio de Biociencia Animal . September 5, Pirassununga, SP, Brazil. 
\vspace{0.5cm}
\item [\bf{11}.] Santana B, {\bf \underline{Mamani GC}}, Oliveira G,  Ventura R, Ferraz JBS. 
Efeitos da endogamia sobre a uniformidade de progenie de touros nelore..
VI Simposio de Biociencia Animal. September 5, Pirassununga, SP, Brazil. 
\end{list1}

\section{\sc 2016}
\begin{list1}
\item [\bf{10}.] {\bf \underline{Mamani GC}}, and Santana B, Aguirre L, Mattos EC, Ferraz JBS. 
Estructura genética de ovinos Santa Inês por análise de pedigree.
III Simposio de Biocióncia Animal . Setiembre 8, Pirassunuga, SP, Brasil. 
\end{list1}

\section{\sc 2015}
\begin{list1}
\item [\bf{9}.] Mamani-Cato RH, Huanca T, Condori-Rojas N,{\bf \underline{Mamani GC}}.
Estructura genética de la población de llamas del Banco de Germoplasma del INIA Perú
Reunión Científica Anual de la Asociación Peruana de Producción Animal. Agosto 7-9, Ayacucho, Perú. 
\vspace{0.5cm}

\item [\bf{8}.] Mamani-Cato RH, Huanca T, {\bf \underline{Mamani GC}} y Condori-Rojas N.
Modelación de curvas de crecimiento de llamas Qara utilizando modelos de crecimiento no lineales
Reunión Científica Anual de la Asociación Peruana de Producción Animal. Agosto 7-9, Ayacucho, Perú. 
\vspace{0.5cm}

\item [\bf{7}.] {\bf \underline{Mamani GC}} y Ferraz JBS. 
Associação entre coeficientes de endogamia genômicos e características reprodutivas em bovinos Nelore.
II Simposio de Biocióncia Animal . Setiembre 11, Pirassunuga, SP, Brasil. 
\vspace{0.5cm}

\item [\bf{6}.] Quina E, Renieri C, Peña Y, {\bf \underline{Mamani GC}}.
Tendencias genéticas para peso de vellón, diámetro y coeficiente de variabilidad de fibra de alpacas del Centro de Desarrollo Alpaquero Toccra
VII Congreso Mundial en Camélidos Sudamericanos. Octubre 28-30, Puno, Perú. 
\vspace{0.5cm}

\item [\bf{5}.] Mamani Torreblanca C, {\bf \underline{Mamani GC}}, Chávez J.
La llama carguera: crónicas contadas por um llamero del Perú
VII Congreso Mundial en Camélidos Sudamericanos. Octubre 28-30, Puno, Perú. 
\end{list1}
\vspace{0.5cm}

\section{\sc 2013}
\begin{list1}
\item [\bf{4}.] Gutiérrez G, Candio J, Ruiz J, {\bf \underline{Mamani GC}}, Corredor A, Flores E.
Performance of alpacas from a dispersed open nucleus in Pasco region, Peru.
64 th EAAP Annual meeting. Symposium on South American Camelids and other Fibre Animals. August 25-30, Nantes, Francia. 
\vspace{0.5cm}

\item [\bf{3}.] {\bf \underline{Mamani GC}}, Huanca T, Gutiérrez G.
Heredabilidad y tendencias genéticas para el peso al nacimiento en alpacas (\textit{Vicugna pacos}) del CIP Quimsachata-INIA, Puno.
XXXVI Reunión Científica Anual de la Asociación Peruana de Producción Animal. Diciembre 4-6, Lima, Perú. 
\end{list1}
\vspace{0.5cm}

\section{\sc 2012}
\begin{list1}
\item [\bf{2}.] Quina E, {\bf \underline{Mamani GC}}.
Índices reproductivos y caracterización fenotípica de llamas (\textit{Lama glama}) en Centro de Desarrollo Alpaquero Toccra - Yanque -Caylloma
VI Congreso Mundial en Camélidos Sudamericanos. Arica, Chile. 
\end{list1}
\vspace{0.5cm}

\section{\sc 2006}
\begin{list1}
\item [\bf{1}.] {\bf \underline{Mamani GC}}, Coila A.
Relación de la morfología, motilidad y vigor espermático de semen de alpacas con la actividad enzimática de las transaminasas GOT y GPT.
IV Congreso Mundial en Camélidos Sudamericanos. Noviembre, Santa María, Salta, Argentina. 
\end{list1}

\vspace{0.9cm}
\section{\sc Participación en Cursos} 
\vspace{2cm}


\section{\sc 2016}
\begin{list2}

\item Curso: "Tópicos Especiais sobre Crescimento e Desenvolvimento Animal". Universidade de São Paulo. Octubre 24 -30. 
Dictado por David Gerrard
\vspace{0.5cm}
\item Curso: "Modelagem Estatistico aplicado ao Melhoramento Animal". Universidade Estadual Paulista UNESP. Novembro 21-25. 
Dictado por Guilherme Rosa. 
\vspace{0.5cm}
\item Curso: "Analise de Dados Genômicos". Universidade Estadual Paulista UNESP. Março- Junho. 
Dictado por Roberto Carvalheiro
\vspace{0.5cm}
\item Curso: "Genética Quantitativa e Genômica" Escola Superior de Agricultura - ESALQ-USP. Mayo 16 - 25.
Dictado por Gota Morota y Matt Splenger.
\end{list2}  
\vspace{0.5cm}


\vspace{0.5cm}
\section{\sc Experiencia en Docencia}



{\bf Facultad de Zootecnia e Ingenieria de Alimentos, Universidad de São Paulo}, SP, Brasil\\\

\vspace{.01pt}
\textbf{Clase teórica}\\
Curso: genética Básica     \hfill {\bf Semestre 2016-II} \\

\vspace{.01pt}
\textbf{Asistente de Profesor} \\
Curso: Estadistica II     \hfill {\bf Semestre 2016-I}\\
\vspace{0.5cm}


{\bf Facultad de Zootecnia, Universidad Nacional Agraria La Molina}, Lima, Perú.\\
\vspace{.01pt}

\textbf{Asistente de Prácticas} \\
Curso: genética de Poblaciones     \hfill {\bf Semestre 2013-II} \\
\vspace{0.5cm}


\section{\sc Habilidades Computación} 
\begin{list2}
\item Lenguaje de programación: Python
\vspace{0.3cm}
\item Programas estadísticos : R, SAS, Asreml
\vspace{0.3cm}
\item Programas de Simulacion: Vensim, QMsim
\vspace{0.3cm}
\item Lenguajes de contenido-descripcion: \LaTeX
\vspace{0.3cm}
\item Sistema Operativo : Windows y Linux 
\end{list2}

\section{\sc Idiomas} 
\begin{list2}
\vspace{0.3cm}
\item Español
\vspace{0.3cm}
\item Quechua
\vspace{0.3cm}
\item Portugués
\vspace{0.3cm}
\item Inglés
\end{list2}

\section{\sc Otros}
\begin{list2}
\item Licencia de conducir carro A1\\
\end{list2}

\section{\sc Referencias}
Estoy disponible a otras referencias y consultas. 

\end{resume}
\end{document}
